%%-------------------------------------------------------------------------
 %
 % api.tex
 %
 %    API of a pbeer. It documents the available events of a pbeer
 %
 % LICENSE
 %
 %    Beernet is released under the Beerware License (see file LICENSE) 
 % 
 % IDENTIFICATION 
 %
 %    Author: Boriss Mejias <boriss.mejias@uclouvain.be>
 %
 %    Last change: $Revision: 403 $ $Author: boriss $
 %
 %    $Date: 2011-05-19 21:45:21 +0200 (Thu, 19 May 2011) $
 %
%%-------------------------------------------------------------------------
 

\documentclass[11pt]{article}

%%-------------------------------------------------------------------------
 %
 % commands.tex
 %
 %    General latex commands for Beernet's documentation.
 %
 % LICENSE
 %
 %    Beernet is released under the Beerware License (see file LICENSE) 
 % 
 % IDENTIFICATION 
 %
 %    Author: Boriss Mejias <boriss.mejias@uclouvain.be>
 %
 %    Last change: $Revision: 403 $ $Author: boriss $
 %
 %    $Date: 2011-05-19 21:45:21 +0200 (Thu, 19 May 2011) $
 %
%%-------------------------------------------------------------------------

\usepackage{geometry}
\usepackage{amsmath}
\usepackage{epsfig}
\usepackage{subfig}
\usepackage{color}
\usepackage{listings}
\usepackage{url}

\graphicspath{{figs/}}

\newcommand{\comment}[1]{}
\newcommand{\ignore}[1]{}
\renewcommand{\labelenumii}{\alph{enumii}}

\definecolor{commentcolor}{gray}{0.4}
\definecolor{keywordcolor}{rgb}{0.0,0.4,0.0}
\lstset{
	emph={for, do, define},emphstyle=\color{keywordcolor}\bfseries,
	language=Oz,
	tabsize=4,
	columns=fullflexible,
	%basicstyle=\footnotesize\fontfamily{ppl},
	basicstyle=\small\fontfamily{ppl},
	keywordstyle=\color{keywordcolor}\bfseries,
	identifierstyle=\color{black}\fontfamily{phv}\selectfont,
	commentstyle=\color{commentcolor}\textit,
	stringstyle=\ttfamily,
	showstringspaces=false
	%numbers=left
	%numberstyle=\tiny
}
\newcommand{\code}[1]{\lstinline[language=Oz]{#1}}

\usepackage[plainpages=false,
			colorlinks,
			citecolor=black,
			linkcolor=blue,
			urlcolor=blue]{hyperref}



\title{Beernet's API}

\author{Boris Mej\'{i}as}

\date{May 31, 2011}

\begin{document}

\maketitle

\section{Introduction}

\begin{quote}

{\bf WARNING!} This document corresponds to the API of Beernet-0.8. It does
not integrate the new API with secrets, but Beernet-0.9 is still compatible with
this document.

\end{quote}

Beernet is provided as a Mozart-Oz\footnote{The Mozart Programming System,
\url{http://www.mozart-oz.org}} library. To use it, the program needs to
import the main functor to create a peer to bootstrap a network, or to join an
existing one using the reference of another peer. Importing the main functor
and creating a new peer works as follows:

\begin{lstlisting}
	functor
	import
		Beernet at 'beernet/pbeer/Pbeer.ozf'
	define
		Pbeer = {Beernet.new args(transactions:true)}
\end{lstlisting}

Interacting with the peer is done by triggering an event as follows:

\begin{lstlisting}
	{Pbeer event(arg1 ... argn)}
\end{lstlisting}

We list now the different events that can be triggered on Beernet's peers.
Even though Beernet's architecture is organized with layers where Trappist is
the upper most one, the architecture does not prevent the access to lower
layers because their functionality is important to implement applications.

\section{Relaxed Ring}

The following events can be used to get access to the functionality provided
by the relaxed ring layer. It mostly provides access to peer's pointers and
other information of the structured overlay network. 

\subsection{Basic Operations}

\begin{itemize}

	\item \code{join(RingRef)} Triggers joining process using \code{RingRef}
as access point.

	\item \code{lookup(Key)} Triggers lookup for the responsible of
\code{Key}, which will be passed through the hash function.

	\item \code{lookupHash(HashKey)} Triggers lookup for the responsible of
\code{HashKey} without passing \code{HashKey} through the hash function.

	\item \code{leave} Roughly quit the network. No gently leave implemented.
 
\end{itemize}

\subsection{Getting Information}

\begin{itemize}

\item \code{getId(?Res)}\\ Binds \code{Res} to the id of the peer

\item \code{getRef(?Res)}\\ Binds \code{Res} to a record containing peer's
reference with the pattern\\ \code{pbeer(id:<Id> port:<Port>)}

\item \code{getRingRef(?Res)}\\ Binds \code{Res} to the ring reference

\item \code{getFullRef(?Res)}\\ Binds \code{Res} to a record containing peer's
reference and ring's reference with the pattern \code{ref(pbeer:<Pbeer Ref>
ring:<Ring Ref>)} 

\item \code{getMaxKey(?Res)}\\ Binds \code{Res} to the maximum key in ring's
address space

\item \code{getPred(?Res)}\\ Binds \code{Res} to the reference of peer's
predecessor

\item \code{getSucc(?Res)}\\ Binds \code{Res} to the reference of peer's
successor

\item \code{getRange(?Res)}\\ Binds \code{Res} to the responsibility range of
peer with the pattern \code{From#To}, where \code{From} and \code{To} are
integers keys.

\end{itemize}

\subsection{Other Events}

\begin{itemize}

\item \code{refreshFingers(?Flag)} Triggers lookup for ideal keys of finger
table to refresh the routing table. Binds \code{Flag} when all lookups are
replied.

\item \code{injectPermFail} Peer stop answering any message. 

\item \code{setLogger(Logger)} Sets Logger as the default service to log
information of the peer. Mostly used for testing and debugging.

\end{itemize}

\section{Message Sending}

This section describe the events that allow applications to send and receive
messages to other peers.

\begin{itemize}

\item \code{send(Msg to:Key)}\\ Sends message \code{Msg} to the responsible of
key \code{Key}.
 
\item \code{dsend(Msg to:PeerRef)}\\ Sends a direct message \code{Msg} to a
peer using \code{PeerRef}.

\ignore{ 
\item \code{broadcast(Msg range:Range)}\\ Sends message \code{Msg} to all
peers on the range \code{Range}, which can be \code{all}, sending to the whole
ring, \code{butMe}, sending to all ring except for the sender, and
\code{From#To}, which sends to all peers having an identifier within keys
\code{From} and \code{To}, so it can be used as a multicast.
}

\item \code{receive(?Msg)}\\ Binds \code{Msg} to the next message received by
the peer, and that it has not been handled by any of Beernet's layer. It
blocks until next message is received. 

\end{itemize}

\section{DHT}

Beernet also provides the basic operations of a distributed hash table (DHT).
None of this uses replication, therefore, there are no guarantees about
persistence.

\begin{itemize}

\item \code{put(Key Val)}\\ Stores the value \code{Val} associated with key
\code{Key}, only in the peer responsible for the key resulting from applying
the hash function to key \code{Key}.

\item \code{get(Key ?Val)}\\ Binds \code{Val} to the value stored with key
\code{Key}. It is bound to the atom \code{'NOT_FOUND'} in case that no value
is associated with such key.

\item \code{delete(Key)}\\ Deletes the item associated to key \code{Key}.

\end{itemize}

\section{Symmetric Replication}

The symmetric replication layer does not provides an interface to store values
with replication, but it does provides some functions to retrieve replicate
data, and to send messages to replica-sets.

\begin{itemize}

\item \code{bulk(Msg to:Key)}\\ Sends message \code{Msg} to the replication
set associated to key \code{Key}.

\item \code{getOne(Key ?Val)}\\ Binds \code{Val} to the first answer received
from any of the replicas of the item associated with key \code{Key}. If value
is \code{'NOT_FOUND'}, the peer does not bind \code{Val} until it gets a valid
value, or until all replicas has replied \code{'NOT_FOUND'}. 

\item \code{getAll(Key ?Val)}\\ Binds \code{Val} to a list containing all
values stored in the replica set associated to key \code{Key}.

\item \code{getMajority(Key ?Val)}\\ Binds \code{Val} to a list containing
the values from the replica set associated to key \code{Key}. It binds
\code{Val} as soon as the majority is reached.

\end{itemize}

\section{Trappist}

Trappist provides different protocols to run transactions on replicated items.
Due to their specific behaviour, they have different interfaces.

\subsection{Paxos Consensus}

\begin{itemize}

\item \code{runTransaction(Trans Client Protocol)}\\ Run the transaction
\code{Trans} using protocol \code{Protocol}. The answer, \code{commit} or
\code{abort} is sent to the port \code{Client}. Currently, the protocols
supported by this interface are \code{twophase} and \code{paxos}, for
two-phase commit and Paxos consensus with optimistic locking. For eager
locking, see the interface in Section~\ref{apx:beernet:eager-locking}.

\item \code{executeTransaction(Trans Client Protocol)}\\ Exactly the same as
\code{runTransaction}. Kept only for backward compatibility.

\end{itemize}

Inside a transaction, there are three operations that can be used to
manipulate data.

\begin{itemize}

\item \code{write(Key Val)}\\ Write value \code{Val} using key \code{Key}. The
new value is stored at least in the majority of the replicas. Updating the
value gives a new version number to the item.

\item \code{read(Key ?Val)}\\ Binds \code{Val} to the latest value associated
to key \code{Key}. Strong consistency is guaranteed by reading from the
majority of the replicas. 

\item \code{remove(Key)}\\ Removes the item associated to key \code{Key} from
the majority of the replicas.

\end{itemize}

\subsection{Paxos with Eager Locking}
\label{apx:beernet:eager-locking}

\begin{itemize}

\item \code{getLocks(Keys ?LockId)}\\ Get the locks of the majority of
replicas of all items associated to the list of keys \code{Keys}. Binds
\code{LockId} to an Oz name if locks are successfully granted, and to
\code{error} otherwise.

\item \code{commitTransaction(LockId KeyValuePairs)} update all items of the
list \code{KeyValuePairs} which must be locked using \code{LockId}. Each
element of the list \code{KeyValuePairs} must be of the form
\code{<key>#<value>}

\end{itemize}

\subsection{Notification Layer}

\begin{itemize}

\item \code{becomeReader(Key)}\\ Subscribes the current peer to be notified
about locking and updates of the item associated to key \code{Key}.
Notification are received using the \code{receive} event described previously.

\end{itemize}

\subsection{Key/Value-Sets}

\begin{itemize}

\item \code{add(Key Val)}\\ Adds the value \code{Val} to the set associated to
key \code{Key}.

\item \code{remove(Key Val)}\\ Removes value \code{Val} from the set associated
to key \code{Key}.

\item \code{readSet(Key ?Val)}\\ Binds \code{Val} to a list containing all
elements from the set associated to key \code{Key}. 

\end{itemize}

\end{document}
